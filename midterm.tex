\documentclass[12pt]{article}
\usepackage{booktabs}
\usepackage{placeins}
\usepackage{natbib}
\usepackage{array}
\usepackage{amsmath}
\usepackage{graphicx}
\usepackage{titling}
\usepackage{setspace} 
\setlength{\droptitle}{-10em}   % This is your set screw
\usepackage{geometry}
\addtolength{\topmargin}{-0.3in}
\addtolength{\textheight}{1.75in}
\doublespacing
\author{Ed Jee} 
\title{Midterm: Dippel 2014}


\begin{document}
% \maketitle


\section{Referee Report}

\cite{dippel}'s primary contribution is to estimate the effect of forced coexistence - whereby autonomous sub-tribal units share a 
reservation - on outcomes such as contemporary income in the context of Native American reservations. This question is of central interest to economists because economic underperformance
today is often attributed to "cleavages" where multiple groups reside within borders that do not reflect historical boundaries. Unfortunately, 
forced coexistence is often endogenous, the author cites examples of tribes with strong reverence/preference for tradition/culture as more likely 
to lobby for a homogeneous reservation. The paper uses a range of data sources from the US Census Bureau, anthropological sources, and historical 
sources describing the sites of mining interest and mining value - in total the author reports 182 observations. The author's preferred specification suggests forced coexistence leads to a $\approx 30\%$ fall in contemporary 
incomes, a substantial negative effect.

\subsection{Selection On Observables}
To overcome concerns of endogeneity the author uses a selection on observables approach followed by IV. In the interest of brevity I'll focus 
on the fixed effects selection on observables strategy over the OLS estimates, Table 2 Panel B, since the removal of unobserved tribe level heterogeneity 
makes causal interpretation of forced coexistence more plausible.  The author regresses\footnote{The author includes historical centralisation ($HC_i$) as an additional 
control since it's of interest to the literature - I ignore it throughout as it's not a primary estimand of the paper.}:
\begin{align*}
    \log(\text{per capita income})_{ie} &= \alpha_e + \gamma_s + \beta_1 FC_{ie} + \beta_2 HC_{ie}+ \beta_3 \text{Res Controls}_{ie} + \beta_4 \text{Tribe Controls}_{ie}  \\
    &+ \beta_5 \text{Extra Res Controls}_{ie} + u_{ie} 
\end{align*}
  
where $\alpha_e, \gamma_s$ are tribe and state fixed effects respectively. We index reservation $i$ and ethnicity $e$, a catchall term for sub-tribal units.
 The interpretation of $\beta_1$ is only causal if we believe $FC_{ie}$ is orthogonal to the error term $u_{ie}$. As the author reports, any selection into 
 $FC$ based off unobservable sub-tribal characteristics will lead to biased estimates and citing such an anecdotal example, motivates the use of an instrument. The author's 
 assessment of the limited value of the FE strategy seems reasonable and after a brief discussion of bad controls we turn to IV.

\subsubsection*{Bad Controls}

A large number of the author's controls are determined after treatment - it seems likely that measures such as log(population), population adult share, and local unemployment are all channels through which forced coexistence can influence contemporary reservation life. It's hard to believe that forced coexistence only influences 
per capita income and has 0 effect on other channels such as unemployment, population demographics etc. The paper should at least caveat this in the selection on observables section 
before discussing IV I believe.

\subsection{Instrumental Variables}
The paper instruments forced coexistence with the value of historical gold and silver mining as separate instruments, and in the preferred specification uses historical mining defined 
as the sum of the two. The author argues the federal government was incentivised to create smaller/fewer reservations, forcing bands to coexist, in areas with higher land value. Since Native 
Americans didn't possess a mining industry, conditional on land characteristics, the presence of valuable minerals increased the attractiveness of land to the federal government but had no
direct impact on Native American value appraisals. Therefore, a higher historical mining value induced the federal government to create fewer reservations and 'pack' multiple autonomous bands
into a single reservation leading to forced coexistence. The author clearly emphasises the threat of direct instrument effects, the value of historical mining  influencing incomes today
and only uses mining value from mines located in ancestral homelands that are outside the border of modern reservations. Whilst this is a good first step ensuring instrument validity, I shall argue 
below that instrument validity, alongside instrument relevance, are still a concern.

\subsubsection*{Instrument Relevance}

Firstly, there is no reason not to report Anderon-Rubin confidence intervals - especially when the first stage $F$-stat is so low. Furthermore, the steps taken in my replication 
(see Section 2) suggest that the gold instrument isn't even significant in contrast to the paper's reported results. Given the presence of weak instruments; and the fact there's only 182 observations it seems prudent to report weak instrument robust CIs. 


\subsubsection*{Instrument Validity and Monotonicity}
The paper is careful to construct the instrument using mines outside contemporary reservation borders however I'd argue this still doesn't fully address instrument validity concerns.

Infrastructure built by the federal/state governments is likely to be driven, at least in part, by a desire to exploit natural resources. Since railroads, roads, and 
cities influence socioeconomic factors for hundreds of miles in their vicinity I'd argue that even after discarding mines within reservation borders and controlling 
for local mining, provided reservations aren't 1000s of miles from ancestral homelands (which doesn't appear to be the case), instrument validity is still a concern.  

Finally, whilst Native Americans weren't able to extract minerals themselves this doesn't mean discovery of minerals didn't influence their appraisal of land value when 
bargaining with the federal government. If larger tribes were able to threaten violence more effectively than smaller tribes, the discovery of minerals could increase their bargaining power and therefore leverage larger reservations and a reduction in forced coexistence - in this case the instrument would have a different sign in the first stage for large vs small tribes and violate instrument monotonicity.


\subsubsection*{Instrument Construction}

The author's preferred specification sums across both gold and silver mining to produce historical mining, this is partially motivated by concerns of week instruments and 
the author justifies his decision using the increase in the first stage $F$-stat. I believe this is a mistake: the two instruments surely contain more information than 
destroying instrument granularity  by pooling information into one instrument - there are many solutions to weak instruments, this is not one of them. Furthermore, the first stage $F$-stat increases due to a mechanical reduction in the degrees of freedom, moving from two to one estimands, that doesn't reflect the true statistical uncertainty of estimated quantities. This mistake is more common in the examiner design literature where leave-out-means use a degree of freedom of 1, and not the number of examiners\footnote{See an unpublished note by \cite{hull} for more details.} when computing weak instrument heuristics.


If the author truly wishes to pool weak instruments he should consider implementing \cite{reqml}'s Random Effects Quasi-Maximum Likelihood estimator which uses data-driven random effects/hierarchical shrinkage to overcome weak instrument issues through pooling. Alternatively, if the author truly believes historical mining 
is his chosen instrument, he should motivate it with theory/historical evidence and not an appeal to poor statistics.


\subsubsection*{Spatial Noise}


Finally, the author follows a strong tradition in economics of regressing contemporary outcomes on historical variables without adjusting for spatial correlation. \cite{persistence} shows that this often leads to inflated $t$-statistics far beyond what seems reasonable. Given the impressive ability of this paper to produce significant results 
with only 182 observations it seems particularly prudent to follow Kelly's advice: Report Moran's I statistic for spatial autocorrelation, and generate spatially correlated noise and measure what fraction of the estimated response is just correlation driven by spatial noise.



Additionally, I'd recommend reporting Conley standard errors with a bandwith as suggested by Kelly. In Section 3 I implement Kelly's advice and show that 
it's plausible the paper's results are in large part driven by spatial autocorrelation.


\subsection{Conclusion}
Given the concerns outlined above regarding instrument weakness, spatial noise, validity, and monotonicity it seems unlikely the author's results genuinely reflect researcher uncertainty 
about the parameter of interest - the effect of forced coexistence on  reservation incomes, leaving aside any discussion of LATE and its interpretation here. I believe it'd be irresponsible to take the proposed estimates of 
-$30\%$ seriously without fixing the problems listed in this report.
\section{Replication} 
% My strategy is as follows:

% \begin{itemize}
%     \item Create controls/instruments etc.
%     \item Create model formulae programmatically.
%     \item Estimate models using the package `fixest'.
%     \item Compare replicated and original estimates programmatically using
%     R's `testthat' library, originally designed for unit tests.  
% \end{itemize}

% The controls are naturally grouped by the author as follows:

% \begin{table}[htbp]
%     \centering\begin{tabular}{| m{4cm} | m{5cm} | m{5cm} | }
%         \hline 
%     Set & Controls  & Variable Name\\ \hline\hline
%     Reservation Controls & log(local per capita  income),
%     log(local unemployment),
%     log(distance to nearest city),
%     log(ruggedness),
%     log(reservation area squared) &
%     \textit{
%         logpcinc\_co,
%         logunempl\_co,
%         logdist,
%         logruggedness,
%         logresarea\_sqkm
%     }  \\  \hline
%     Tribe Controls & subsistence patterns,
%     sedentariness,
%     wealth distinctions,
%     social complexity &
%     \textit{
%         ea\_v5,
%         ea\_v30,
%         ea\_v32,
%         ea\_v66
%     }\\ \hline
%     Extra Reservation Controls & log(population),
%     log(population)$^2$,
%     adult population share,
%     casino present &
%     \textit{
%         logpop,
%         logpopsq,
%         popadultshare,
%         casion
%     }\\ \hline
%     IV controls & ancestral ruggedness,
%     distance from ancestral lands,
%     local historical mining value (gold \& silver  ) 
%     & \textit{
%         homelandruggedness,
%         removal,
%         wgold\_enviro,
%         wsilver\_enviro
%     }\\ 
%     \hline
    
%     \end{tabular}
%     \caption{Control List}
%     \label{controls}
% \end{table}

\subsection{ OLS and Fixed Effects}
First, I replicate the OLS and FE tables using the estimating equations:
\begin{align*}
    \log(p.c. \ income)_i &= \alpha + \beta_1 \text{Forced coexistence}_i   + \beta_2 \text{Historical centralisation}_i + X_i \Gamma  + \varepsilon_i
\end{align*}
 where $X_i$ corresponds to a matrix of covariates described by the author. The 
 fixed effects model is identical except we control for tribe (and sometimes state) fixed effects. Standard 
 errors are clustered at the tribe and state level. I omit the OLS estimates here for brevity
 and only display the FE estimates. 
% \input{data/output/table-2-ols.tex}




% \input{data/output/table-4-A.tex}
% \input{data/output/table-4-B.tex}


\input{data/output/table-2-fe.tex}


Both OLS and FE replicated estimates are identical to Dippel (2015) estimates 
after rounding to three decimal places. $t$-statistics are similarly identical apart 
from column 5. Regardless, the estimates are still significant and there's no material difference in estimates.

\subsection{Instrumental Variables}
I estimate the following:
\begin{align*}
    \log(p.c. \ income)_i &= \alpha + \beta_1 \widehat{\text{Forced coexistence}}_i   + \beta_2 \text{Historical centralisation}_i + X_i \Gamma  + \varepsilon_i \\ \\
\text{Forced coexistence}_i &= \pi_0 + \pi_1 \text{Historical gold-mining}_i + \pi_2 \text{Historical silver-mining} + X_i \Pi_3 + u_i
\end{align*}
Where historical gold and silver mining are our instruments for the endogenous variable "Forced coexistence".

IV estimates replicate less cleanly: columns 1, 2, and 3 replicate perfectly; columns 4, 5, and 6 can be a little different, although at first glance not sizably different. The author 
adds sets of controls in identical stages as in the OLS and FE estimation strategies and the number of observations is identical. Since columns 1-3 
replicate perfectly and column 4 fails it seems likely there's a discrepancy between the additional reservation controls the author defines during OLS
and IV. I omit Panel A in the interest of space as Dippel describes the single instrument specification as his preferred. Comparing the final column in the preferred specification 
we see a difference of $-0.486 - - 0.443 = -0.043 \approx 10\%$ difference.

% \input{data/output/table-5-iv-A.tex}
\input{data/output/table-5-iv-B.tex}

%  The discrepancy in columns 4,5, and 6 continue in the first stage and reduced form breakdowns in Table 3 (omitted here for brevity). This is
% somewhat worrying since this leads to the gold mining instrument becoming insignificant in specifications 4 and 6 and a further reduction in an already small 
% first stage $F$-statistic. 
% \FloatBarrier
\section{Extensions}
\subsection{Anderson-Rubin}

First, I estimate Anderson-Rubin weak IV robust confidence intervals since the first stage $F$-stats reported in the author's preferred 
specification are still somewhat low with the majority of columns 1-5 in Table 3 (Dippel Table 5) barely above the common heuristic of 10. Even
without conditioning on $F$-stat pre-tests computing weak IV robust confidence intervals is sensible.

\begin{figure}[htbp]
    \centering
    \includegraphics[scale=0.3]{data/output/anderson-rubin-plot.png}
    \caption{Anderson Rubin vs Original Confidence Intervals}
    \label{AR}
\end{figure}
In Figure \ref{AR} I plot AR confidence intervals alongside the author's conventional clustered standard error confidence intervals using the Dippel Table 5, Panel B specification (IV - one instrument).
Clearly, this leads to a large increase in the CI width and all but the final estimate are now insignificant from 0. We should doubt somewhat the author's primary conclusion as the results
do not appear to be robust to weak instruments.
\subsection{Counterfactual Densities}
In order to explore potential outcome densities using the results of \cite{imbens-rubin}
I discretise the instrument and treatment to create binary variables where each variable is set to $1$ if the realisation is above its
state median - using means instead of medians leads to only $\approx 3$ compliers due to the fat tailed nature of the data, as opposed to $\approx 35$ using medians.


\begin{figure}[htbp]
    \centering
    \includegraphics[scale=0.3]{data/output/density-plot.png}
    \caption{Density Plots}
    \label{dens}
\end{figure}


The density for $Y_1$ is clearly shifted to the left of the potential outcome density $Y_0$ which makes sense given the negative effect 
of forced coexistence estimated by the author\footnote{Although given he uses the instrument in its continuous, not binary form, these LATEs aren't directly comparable}. Interestingly, the $Y_1$ density is almost uniformly to the left of the $Y_0$ density i.e. 
every quantile of the distribution performs worse under treatment although we cannot make any statements about individuals without 
further assumptions such as rank preservation. The estimated densities for always- and never-takers are somewhat different to the compliers 
and it seems likely we can't extrapolate the LATE estimate to other sub-populations.
% \subsection{Additional Robustness Checks}

% Since the paper's main finding doesn't seem to be robust to weak instruments I explore additional robustness checks using 
% \cite{amip}'s Automatic Finite-Sample Robustness Metric. This identifies which in-sample data points have maximal influence on 
%  metrics of interest such as the significance or sign of a result. Whilst perfectly identifying these points is computationally infeasible
% since it involves iterating through the power set of all data points and deleting $k = 1, 2, 3, ...$ data points the author's calculate approximate 
% maximal influence in one step using a Taylor expansion.



% \input{data/output/amip.tex}

% The robustness checks again suggest we should be cautious interpretting the author's primary result - deleting only 7 data points leads to insignificance and 
% removing 19 completely reverses the sign of the estimate.

\subsection{Spatial Noise}
Finally, I investigate whether the author's estimates could merely be the result of fitting spatial noise. \cite{persistence} shows that many modern 
papers in economics exhibit unusually large $t$-statistics due to regressing modern outcomes on historical characteristics without adequately accounting for 
spatial noise. 

% Kelly suggests two solutions\footnote{He notes that commonly used Conley standard errors often use a bandwidth magnitudes too small to account for reasonable levels of spatial correlation.}: Reporting Moran's I statistic, a two dimensional analog of the Durbin-Watson statistic,
%  and  generating spatial noise to replace the dependent variable and treatment variable in turn followed by measuring how much variation in outcomes can simply be explained by spatially correlated noise.

\subsubsection*{Moran's I}
Unfortunately, whilst the dataset provided contains a `geoid2' corresponding to Census Bureau identifiers there's not enough information in the paper to map these 
into locations. Instead, I use the OpenStreetMap API \citep{OpenStreetMap} to identify reservation locations based off a string column containing reservation names. This leads 
to 156 matches and  26 unidentified locations. Using the remaining data I estimate Moran's I for our dependent variable using an MC simulation of 100,000 draws\footnote{Calculating Moran's analytically can be sensitive to irreguarly distributed polygons - i.e. US state/reservation boundaries.}.
% \begin{figure}[htbp]
%     \centering
%     \includegraphics[scale=0.3]{data/output/moran-plot.png}
%     \caption{Moran's I}
%     \label{moran}
% \end{figure}

We soundly reject the null hypothesis of no spatial autocorrelation. The $p$-value associated with the test is $0.00009$.


\subsubsection*{Spatial Noise Simulations}

To create spatial noise I generate a Gaussian process with covariance kernel:
\begin{align*}
    \Sigma(x_i, x_j) &= \exp(- || x_i - x_j ||^2) 
    = \Sigma_D
\end{align*}
i.e. the inverse of the squared Euclidean distance between two reservations. I sample $1000 \times 156$ draws from the corresponding 
Gaussian process:
\begin{align*}
    Y \sim MVN(\mathbf{0}_{156}, \Sigma_D)
\end{align*}


Geostatistics often uses  \textit{kriging} to interpolate between points - this is identical to using a Gaussian process and motivates 
my use of the Gaussian here. Since the covariance kernel is a function of the proximity of points any random draws will naturally exhibit spatial 
correlation and the exponential ensures correlation dies off as proximity falls. An example draw is shown in Figure \ref{noise} where lighter, larger 
points correspond to higher simulated values.


\begin{figure}[htbp]
    \centering
    \includegraphics[scale=0.3]{data/output/noise-plot.png}
    \caption{Spatially Correlated Noise}
    \label{noise}
\end{figure}

Finally, I estimate the author's preferred IV specification using noise as the outcome variable and record the $p$-values. Since any noise 
was generated randomly with respect to the instrument/endogenous variable the only channel through which correlation can occur is through spatial 
autocorrelation \textit{despite the fact this spatial noise was generated without using any information in the covariates, dependence purely arises through 
the distance metric in the covariance kernel}. That is, if the treatment/instrument is spatially autocorrelated we can generate seemingly significant results 
even when fitting noise. Below is the quantile-quantile plot of the $p$-values - under the null hypothesis i.e. $\beta_{2SLS} = 0$ we'd expect to see $p$-values 
distributed uniformly and follow the 45 degree line.   

\begin{figure}[htbp]
    \centering
    \includegraphics[scale=0.3]{data/output/qq-plot.png}
    \caption{Simulated $p$-value QQ plot}
    \label{qq}
\end{figure}

Figure \ref{qq} clearly shows the simulated $p$-values depart substantially from uniformity and it seems likely that Dippel 2014's results are at least 
in part driven by fitting spatial noise.



\FloatBarrier
\bibliographystyle{abbrvnat}
\bibliography{bib}
\end{document}